\documentclass[11pt,a4paper]{article}
\usepackage[hyperref]{acl2018}
\usepackage{times}
\usepackage{latexsym}
\usepackage{url}
\usepackage{multirow}
\usepackage{amssymb}
\usepackage{amsmath}
\usepackage{hyperref}
\usepackage{algorithm}
\usepackage{algpseudocode}
\usepackage{graphicx}

%\def\aclpaperid{***} %  Enter the acl Paper ID here

\title{Supplemental Material to: \\
Toward Featureless Event Coreference Resolution via Conjoined Convolutional Neural Networks}
\begin{document}
\maketitle

\section{Learned Parameters for the CCNN Model, per the Dev Set}
\begin{itemize}
\item Features: Lemma and Character Embeddings
\item \# Training Epochs: 20
\item Context Window Size: 0 (this agrees with the Choubey's, et. al. findings \shortcite{Choubey2017EventCR}).
\item \#Negative Examples per Positive (training): 5
\item Batch Size: 128
\item Pool Type: MaxPooling
\item Word Embeddings (used for Lemma Embeddings): GloVe, 300 dimension, trained on 6 Billion Token Corpus
\item Dropout: 0.0
\item Optimizer: Adam
\item \#Kernels: 64 (at every level)
\end{itemize}

\section{Learned Parameters for the Neural Clustering (NC) Model, per its own Dev Set}
\begin{itemize}
\item \# Hidden Units: 50
\item Batch Size: 5
\item \#Negative Examples per Positive (training): 5
\item Initializer: Normal
\item Optimizer: Adam
\item Learning Rate: 0.001
\end{itemize}

\section{Training / Development / Test Data}
Other researchers who have used the ECB+ corpus use Topics 1-22 for training, 23-25 as development, and 26-45 as testing.  We adhere to the same; however, since our NC model needs CCNN's predictions as training data, we must remove some of the training data and use it for separate development sets instead.  Our ful
\bibliography{acl2018}
\bibliographystyle{acl_natbib}
\end{document}