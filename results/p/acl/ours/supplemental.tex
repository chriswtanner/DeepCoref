\documentclass[11pt,a4paper]{article}
\usepackage[hyperref]{acl2018}
\usepackage{times}
\usepackage{latexsym}
\usepackage{url}
\usepackage{multirow}
\usepackage{amssymb}
\usepackage{amsmath}
\usepackage{hyperref}
\usepackage{algorithm}
\usepackage{algpseudocode}
\usepackage{graphicx}
\usepackage{tikz}
\def\checkmark{\tikz\fill[scale=0.4](0,.35) -- (.25,0) -- (1,.7) -- (.25,.15) -- cycle;}
\def\aclpaperid{1355} %  Enter the acl Paper ID here

\title{Supplemental Material to: \\
Toward Featureless Event Coreference Resolution via Conjoined Convolutional Neural Networks}
\begin{document}
\maketitle

\section{Learned Parameters for the CCNN Model, per the Dev Set}
\begin{itemize}
\item Features: Lemma and Character Embeddings
\item \# Training Epochs: 20
\item Context Window Size: 0 (this agrees with the Choubey's, et. al. findings \shortcite{Choubey2017EventCR}).
\item \#Negative Examples per Positive (training): 5
\item Batch Size: 128
\item Pool Type: MaxPooling
\item Word Embeddings (used for Lemma Embeddings): GloVe, 300 dimension, trained on 6 Billion Token Corpus
\item Dropout: 0.0
\item Optimizer: Adam
\item \#Kernels: 64 (at every level)
\end{itemize}

\section{Learned Parameters for the Neural Clustering (NC) Model, per its own Dev Set}
\begin{itemize}
\item \# Hidden Units: 50
\item Batch Size: 5
\item \#Negative Examples per Positive (training): 5
\item Initializer: Normal
\item Optimizer: Adam
\item Learning Rate: 0.001
\end{itemize}

\section{Training / Development / Test Data}
Other researchers who have used the ECB+ corpus use Topics 1-22 for training, 23-25 as development, and 26-45 as testing.  We adhere to the same; however, since our NC model needs CCNN's predictions as training data, we must remove some of the training data and use it for separate development sets instead.  Our full splits are shown in Table \ref{tab:splits}.

% Please add the following required packages to your document preamble:
% \usepackage{multirow}
\begin{table*}[h]
\centering
\begin{tabular}{lr|c|c|c|c|c|c|c|}
\cline{3-9}
\multicolumn{1}{c}{}                           & \multicolumn{1}{l|}{} & \multicolumn{7}{c|}{ECB+ Topics}          \\ \cline{3-9} 
\multicolumn{1}{c}{}                           & \multicolumn{1}{l|}{} & 1-18 & 19 & 20 & 21 & 22 & 23-25 & 26-45 \\ \hline
\multicolumn{1}{|c|}{\multirow{3}{*}{CCNN}}    & TRAIN                 & \checkmark    &    &    &    &    &       &       \\ \cline{2-9} 
\multicolumn{1}{|c|}{}                         & DEV                   &      &    &    &    &    & \checkmark     &       \\ \cline{2-9} 
\multicolumn{1}{|c|}{}                         & PREDICT               &      & \checkmark  & \checkmark  & \checkmark  & \checkmark  &       & \checkmark     \\ \hline
\multicolumn{1}{|l|}{\multirow{3}{*}{NC (WD)}} & TRAIN                 &      & \checkmark  &    &    &    &       &       \\ \cline{2-9} 
\multicolumn{1}{|l|}{}                         & DEV                   &      &    & \checkmark  &    &    &       &       \\ \cline{2-9} 
\multicolumn{1}{|l|}{}                         & PREDICT               &      &    &    & \checkmark  & \checkmark  &       & \checkmark     \\ \hline
\multicolumn{1}{|l|}{\multirow{3}{*}{NC (CD)}} & TRAIN                 &      &    &    & \checkmark  &    &       &       \\ \cline{2-9} 
\multicolumn{1}{|l|}{}                         & DEV                   &      &    &    &    & \checkmark  &       &       \\ \cline{2-9} 
\multicolumn{1}{|l|}{}                         & PREDICT               &      &    &    &    &    &       & \checkmark     \\ \hline
\end{tabular}
\caption{The data splits used by each of our models.  Note, ``predict'' means that we use the model to make predictions on the given data.}
\label{tab:splits}
\end{table*}


\bibliography{acl2018}
\bibliographystyle{acl_natbib}
\end{document}